\documentclass[11pt]{article}

\usepackage[a4paper, top=2.2cm, bottom=2.5cm, left=2.5cm, right=2.5cm]{geometry}

\usepackage{titlesec}
\titleformat{\section}
  {\normalfont\fontsize{20}{10}\bfseries}{\thesection.}{0.5em}{}
\titleformat{\subsection}
  {\normalfont\fontsize{15}{15}\bfseries}{\thesubsection.}{0.5em}{}


  
\usepackage[colorlinks=true,linkcolor=black,urlcolor=blue,bookmarksopen=true]{hyperref}
\usepackage{bookmark}
\usepackage{enumitem}
\usepackage{fancyhdr}
\usepackage[english]{babel}
\usepackage[utf8]{inputenc}
\usepackage[T1]{fontenc}
\usepackage{graphicx}
\usepackage{setspace}
\usepackage{listings}
\usepackage{amsmath, amssymb, amsfonts}
\usepackage{url}
\usepackage{fancyvrb}
\usepackage{caption}
\graphicspath{{img/}}
\pagestyle{fancy}
\fancyhf{}
\fancyfoot[C]{\thepage}
\renewcommand{\headrulewidth}{0pt}
\renewcommand{\footrulewidth}{0.4pt}
\makeatletter
\renewcommand{\footrule}{
  \vskip12pt
  \hrule width\headwidth height\footrulewidth \vskip\footruleskip
}
\makeatother
\makeatletter
\renewcommand{\numberline}[1]{#1.\hskip0.7em}
\makeatother


\usepackage{parskip}
\setlength{\parskip}{10pt}
\setlength{\parindent}{0pt}

\begin{document}
\begin{titlepage}
  \begin{center}
    \vspace*{2cm}
    \includegraphics[width=0.65\textwidth]{login-logo.png}
    \vskip3cm
    {\Huge\bfseries Optimistic and Zero-Knowledge Rollups}\\[1em]
    {\Large\bfseries Understanding blockchain}\\[10em]
    {\large Angelo PICERNO }\\[1em]
    {\large Eduardo TEIXEIRA DE SOUSA }\\[1em]
    {\large Elena PEROTTI }\\[1em]
    {\large Micaela MASRI }\\[1em]
    {\large Adrien SEIGLE }\\[4em]
    {\small November 2025}
  \end{center}
  \vfill
\end{titlepage}

\begingroup
\setstretch{1.25}
\tableofcontents
\thispagestyle{empty}
\endgroup


\begin{spacing}{1.5}
\newpage
\section{Introduction}
The emergence and widespread adoption of blockchain technology has revealed a fundamental structural limitation: scalability. Along with decentralization and security, scalability is one of the three fundamental components of the blockchain trilemma. The trilemma states that a blockchain system can only optimize two of these three properties at the same time, forcing designers to accept trade-offs in performance or trust assumptions.

\begin{figure}[h]
    \centering
    \includegraphics[width=0.4\linewidth]{trilemma.jpg}
    \caption{The blockchain trilemma}
    \label{fig:drawio}
\end{figure}

Interestingly, scalability challenges are not unique to blockchain systems. Traditional database architectures have long faced a comparable tension, often described as the CAP Theorem. Even though applied to different technologies, both frameworks highlight the same underlying idea: achieving perfect performance, resilience and correctness at the same time is structurally difficult, and often impossible.

Blockchain scalability can thus be understood not as an isolated limitation, but as the continuation of a broader historical challenge in distributed systems.

There are two main strategies to improve blockchain scalability. The first one is to modify the underlying blockchain architecture (Layer1, or L1). The second is to implement a second layer (L2) consisting of auxiliary technologies built on top of L1.

Among L2 solutions, Rollups (and in particular Optimistic Rollups and Zero‑Knowledge Rollups) have gained central importance because they shift most computational load
off‑chain while preserving the security guaranteed by Layer 1.


This document examines the technical and functional characteristics of both approaches and discusses their differences, advantages and limitations, as well as potential adoption scenarios.
The goal is to assess how these mechanisms contribute to addressing the scalability challenge and to evaluate the extent to which they can offer a viable path toward more efficient and widely deployable blockchain systems.

\section{Technical Background}
\section{Optimistic Rollups}
\section{Zero Knowledge Rollups}
\section{Comparison Optimistic vs ZK Rollups}
\section{Conclusion}
The analysis of Optimistic Rollups and Zero-Knowledge Rollups has highlighted how both solutions represent fundamental tools for overcoming the scalability limits of Ethereum and, more generally, of public blockchains.
Optimistic Rollups, based on dispute mechanisms (fraud proofs), offer a high level of compatibility with the EVM and guarantee security through cryptoeconomic incentives, but entail longer finalization times due to the challenge periods.
ZK-Rollups, on the other hand, thanks to the use of cryptographic proofs (validity proofs), ensure immediate transaction finality and greater efficiency in resource management, although they require more specialized hardware and may involve a higher risk of operator centralization.
In general, both solutions contribute to making blockchains more scalable, reducing transaction costs and times without compromising the security and decentralization guaranteed by Layer 1. The choice between Optimistic and ZK-Rollups depends on specific needs regarding scalability, speed, and computational complexity, as well as the application context.
In the future, the evolution of Rollups, integration with sharding techniques, and improvements in validity proofs could make these solutions even more performant and accessible, paving the way for a new generation of decentralized applications with high transaction throughput and for large-scale adoption of blockchain in real-world contexts.

\end{spacing}
\end{document}
